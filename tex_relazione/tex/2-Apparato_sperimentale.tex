Lo scopo della prova è misurare la caratteristica I-V di due diodi in silicio e in germanio col fine di verificare la validità dell'\autoref{eq:shokley}, mediante la stima dei parametri $I_0$ e $\eta V_T$.

\subsection{Materiali e strumenti}

Riportiamo adesso un elenco dei materiali e degli strumenti utilizzati durante l’esperienza (tra parentesi il modello utilizzato):
\begin{multicols}{2}
    \begin{itemize}
        \item Generatore di tensione (\verb|TTi EB2025T|)
        \item Potenziometro da 1\si{\kilo\ohm}
        \item Diodo al silicio (\verb|1N914A|/\verb|1N4446|/\verb|1N4148|)
        \item Diodo al germanio (\verb|AAZ15|/\verb|OA47|)
        \item Breadboard
        \item Oscilloscopio analogico (\verb|GW GOS-652|)
        \item Multimetro digitale (\verb|ISO-TECH IDM 105|)
        \item [\vspace{\fill}]
    \end{itemize}
\end{multicols}
\noindent Le specifiche del multimetro sono riportate nella \autoref{tab:multimetro}. La precisione dell'oscilloscopio è del 3\% come dichiarato dal costruttore.

\begin{table}[h]
    \centering
    \begin{tabular}{||c|c|c|c||}
        \hline\hline
        \multicolumn{4}{||c||}{\textsc{corrente (DC)}}\\
        \hline\hline
        fondo scala & risoluzione & prec. & digits \\\hline
        4, 40, 400 \si{\milli\ampere} & 1, 10, 100 \si{\micro\ampere} & 0.4\% & 2\\
        10 \si{\ampere} & 10 \si{\milli\ampere} & 0.8\% & 4\\
        \hline\hline
        \multicolumn{4}{||c||}{\textsc{tensione (DC)}}\\
        \hline\hline
        fondo scala & risoluzione & prec. & digits \\\hline
        400 \si{\milli\volt} & 0.1 \si{\milli\volt} & 0.75\% & 2\\
        4, 40, 400, 1000 \si{\volt} & 1, 10, 100, 1000 \si{\milli\volt} & 0.5\% & 2\\
        \hline\hline
    \end{tabular}
    \caption{Specifiche tecniche del multimetro \emph{\texttt{ISO-TECH IDM 105}} relative alla corrente e alla tensione in corrente continua. La colonna \emph{prec.} indica la precisione e \emph{digits} indica il numero da aggiungere all'ultima cifra significativa della misura.}
    \label{tab:multimetro}
\end{table}


\subsection{Svolgimento dell'esperienza}
\begin{figure}[htb]
    \centering
    \begin{subfigure}{.45\textwidth}
    \centering
        \begin{circuitikz}[scale = 1, every node/.style={scale=1}]
\draw
    (0.5,0)
    -- (0.5,6)
    -- (2,6)
    -- (2,5)
    to[battery2, l_=5\si{\volt}, invert] (2,4)
    -- (2,2)
    to[pR, l_=1\si{\kilo\ohm}, name=P, wiper pos = 0.8] (2,0.5)
    -- (2,0)
    -- (0.5,0)
;
\draw % dalla base della freccetta
    (P.wiper)
    -| (4.5,1.5)
;
\draw % opposto al potenziometro
    (2,5.1)
    to[short, *-] (4.5,5.1) 
    -- (4.5, 4.5)
;
\draw % parte interna
    (4.5,1.5)
    -- (6,1.5)
    to[smeter, t=V, l=$V_\text{MM}$] (6,4.5)
    -- (3,4.5)
    to[oscope, l=$V_\text{osc}$] (3,1.5)
    -- (4.5,1.5)
;
\end{circuitikz}
        \caption{Schema del circuito utilizzato per la calibrazione dell'oscilloscopio.}
        \label{fig:circ_cal}
    \end{subfigure}
    \hspace{1cm}
    \begin{subfigure}{.45\textwidth}
        \centering
        \begin{circuitikz}[scale = 1, every node/.style={scale=1}]
\draw
    (0.5,0)
    -- (0.5,6)
    -- (2,6)
    -- (2,5)
    to[battery2, l_=5\si{\volt}, invert] (2,4)
    -- (2,2)
    to[pR, l_=1\si{\kilo\ohm}, name=P, wiper pos = 0.8] (2,0.5)
    -- (2,0)
    -- (0.5,0)
;
\draw % ramo fuori (dalla base della freccetta)
    (P.wiper)
    -| (5,1.5)
    %to[smeter, t=A, l_=$I$] (5,2.9)
    %-- (5,2.8)
    to[full diode] (5,4.5)
    -- (5,5.1)
    to[smeter, t=A, l_=$I$, -*] (2,5.1)
;
\draw % ramo dell'oscilloscopio (dall'alto)
    (5,4.2)
    to[short, *-] (3.5,4.2)
    to[oscope, l_=$V_\text{osc}$] (3.5,1.7)
    to[short, -*] (5,1.7)
;
\end{circuitikz}
        \caption{Schema del circuito realizzato per la misura della caratteristica I-V dei diodi.}
        \label{fig:circ_diodi}
    \end{subfigure}
    \caption{I due circuiti realizzati nello svolgimento dell'esperienza. $V_\text{osc}$ indica l'oscilloscopio; $I$ e $V_\text{MM}$ indicano il multimetro.}
    \label{fig:circuiti}
\end{figure}

La calibrazione è stata effettuata per verificare la compatibilità degli strumenti di misura. In \autoref{fig:circ_cal} è rappresentato il circuito utilizzato per effettuare la calibrazione dell’oscilloscopio tramite il multimetro digitale. Il circuito è costituito da un generatore che eroga una tensione di 5\si{\volt} collegato ai pin fissi del potenziometro. Sono presenti, inoltre, l’oscilloscopio analogico e il multimetro digitale come rappresentato in \autoref{fig:circuiti}.
Per effettuare la calibrazione, dopo aver impostato lo zero dell’oscilloscopio, è stato necessario variare il potenziometro al fine di ottenere il valore di tensione desiderato con l’oscilloscopio, cambiando fondo scala a seconda della misura, per poi poterlo confrontare con quello misurato dal multimetro. In seguito, abbiamo effettuato un fit lineare del tipo:
\begin{equation} \label{eq:cal}
    V_\text{osc} = a + b \cdot V_\text{MM}
\end{equation}
Per poter considerare compatibili le misure, il risultato del fit deve essere compatibile entro gli errori con una retta inclinata di $45^\circ$ e con intercetta nulla.

In \autoref{fig:circ_diodi} è rappresentato il circuito utilizzato per studiare la caratteristica I-V di un diodo, prima al silicio e poi al germanio. Esso è costituito dal generatore di tensione costante di 5\si{\volt} collegato ai pin fissi del potenziometro, mentre il pin variabile di quest’ultimo è collegato in serie al multimetro e al diodo. È presente, inoltre, come possibile osservare in figura, l’oscilloscopio per misurare il valore di tensione ai capi del diodo $V_D$, ottenuti variando il potenziometro. Per misurare la corrente I è stato invece utilizzato il multimetro. Infine, effettuando un fit sui dati ottenuti dei diodi al silicio e al germanio abbiamo stimato i parametri $I_0$ e $\eta V_T$ secondo l'\autoref{eq:shokley}. 