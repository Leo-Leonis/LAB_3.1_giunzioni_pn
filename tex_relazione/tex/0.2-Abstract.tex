Lo scopo dell’esperienza è stato quello di misurare le caratteristiche I-V di due diodi a semiconduttore costituiti da materiali diversi, silicio (Si) e germanio (Ge), in polarizzazione diretta. In particolare, acquisendo e analizzando i dati relativi alla tensione ai capi di questi ultimi e alla corrente, è stato possibile ricavare i valori dei parametri $I_0$ e $\eta V_T$, presenti nell’Equazione di Shockley che descrivono le caratteristiche fisiche di un diodo. Per quanto riguarda il diodo al silicio, i valori dei parametri trovati sono rispettivamente $I_0^\text{Si} = (1.6 \pm 1.3)$ \si{\nano\ampere} e $\eta^\text{Si}V_T = (47 \pm 4)$ \si{\milli\volt}. Per il diodo al germanio abbiamo invece $I_0^\text{Ge} = (1.8 \pm 1.4)$ \si{\micro\ampere} e $\eta^\text{Ge}V_T = (37 \pm 9)$ \si{\milli\volt}. I due valori di $I_0$ rientrano nell’ordine di grandezza atteso; il valore di $\eta V_T$ del diodo al silicio è compatibile, mentre quello del germanio presenta una piccola discrepanza rispetto alle previsioni iniziali.