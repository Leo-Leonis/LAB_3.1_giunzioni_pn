\usepackage{amsmath}
\usepackage{amscd}
%\numberwithin{equation}{subsection}
\usepackage{amssymb}
\usepackage{amsthm}
\usepackage{appendix}
\usepackage[italian]{babel} % lingua di scrittura
\usepackage{bm} % per avere simboli matematici in grassetto
\usepackage{blindtext} % per generare "Lorem ipsum dolor..." con \blindtext
\usepackage{booktabs} % per le linee nelle tabelle migliori
\usepackage{cancel} % per cancellare un testo \cancel{} per una barra verso alto a dx, mentre \xcancel{} per una croce
\usepackage{csvsimple-l3} % per convertire csv in tabelle
\usepackage{enumitem} % per fare una lista numerizzata
%\usepackage{eurosym} % per il simbolo dell'euro "€" più carino
%\usepackage{extpfeil} % per avere frecce più lunghe
% \usepackage{fontspec} % per il font calibri (NB lo spazio dopo "Calibri")
%     \setmainfont{Calibri }[
%         Path = ./fonts/,
%         Extension = .ttf,
%         BoldFont = *Bold,
%         ItalicFont = *Italic,
%         UprightFont = *Regular
%     ]
\usepackage{geometry} % per la geometria del foglio
    \geometry{
        a4paper,
        total={170mm,247mm},
        left=20mm,
        top=30mm,
    }
\usepackage{graphicx} % per le immagini
\usepackage{wrapfig} % per immagini incorniciate con il testo
\usepackage{geometry}
\usepackage{indentfirst} % per indentare anche il primo paragrafo
\usepackage[hidelinks,colorlinks = true, linkcolor = Blue]{hyperref} % per gli hyperlink
\usepackage[utf8]{inputenc} % molto importante
%\usepackage{minted} % per codice (https://it.overleaf.com/learn/latex/Code_Highlighting_with_minted)
\usepackage{multicol} % per avere parti del testo a più colonne
    % \setlength{\columnseprule}{0.5pt} % Separator ruler width
    % \def\columnseprulecolor{\color{gray}} % Separator ruler colour
\usepackage{multirow} % per avere più colonne unite nelle tabelle
\usepackage{siunitx} % per avere i font per unità internazionali (per esempio microsecondi usare in TEXT MODE "\unit{\micro\second}")
\usepackage{url}
\usepackage[svgnames]{xcolor}

% \usepackage{setspace} % per l'interlinea
% \usepackage[skip=0pt plus2pt, indent = 30pt]{parskip} % spaziatura tra paragrafi
%     %\setlength{\parindent}{15pt}
%     \singlespacing %\doublespacing e \onehalfspacing

\usepackage{tikz}
\usepackage[most]{tcolorbox} % per i testi con la cornice (più info su https://ctan.mirror.garr.it/mirrors/ctan/macros/latex/contrib/tcolorbox/tcolorbox.pdf)
    %\tcbuselibrary{skins}
    \newtcolorbox[]{mvpeqbox}[1]{ % [1] sta per "1 variab. di input
                    title = \textit{{#1}}, % {#1} è la variabile inserita obbligatoria
                    sharp corners, % round corners
                    %colframe = red!50!white, % colore cornice
                    %colback = red!50!white, % colore sfondo
                    boxrule = -0.5pt, % spessore cornice
                    %opacityframe = 0, % opacità cornice
                    %opacityback = 1, % opacità sfondo
                    colbacktitle = white!87!black, % colore sfondo del titolo
                    coltitle = blue, % colore font titolo
                    ams align, % per avere l'ambiente di align
    } % custom tcolorbox per i blocchi di normali equazioni
    \newtcolorbox[]{eqbox}[1]{ % [1] sta per "1 variab. di input
                    title = #1, % #1 è la variabile inserita
                    sharp corners, % round corners
                    colframe = white!95!black, % colore cornice
                    colback = white, % colore sfondo
                    boxrule = 00.75pt, % spessore cornice
                    %opacityframe = 0, % opacità cornice
                    %opacityback = 1, % opacità sfondo
                    colbacktitle = white!95!black, % colore sfondo del titolo
                    coltitle = black, % colore font titolo
                    ams align, % per avere l'ambiente di align
    } % custom tcolorbox per i blocchi di equazione IMPORTANTI
    \newtcolorbox[]{mvvpeqbox}[1]{
                    title = \textit{{#1}},
                    sharp corners, % round corners
                    boxrule = -0.5pt, % spessore cornice
                    colbacktitle = white!87!black, % colore sfondo del titolo
                    coltitle = blue, % colore font titolo
                    ams align,
    } % custom tcolorbox per i blocchi di normali equazioni
\usepackage{circuitikz} % per realizzare circuiti con tikz(fa l'updload automatico del pacchetto tikz) 
%   \usetikzlibrary{patterns,plotmarks} % per usare root con tikz

\usepackage{fancyhdr} % per i headings e footers
    \pagestyle{fancy}
    \fancyhf{}
    \rhead{\textsc{Giovanni Boccia}, \textsc{Leandro Ye}}
    \lhead{Caratteristiche I-V di diodi al silicio e al germanio}
    \fancyfoot[c]{\thepage}

\usepackage{caption} % per avere didascalie migliori
    \captionsetup{width=0.9\textwidth} % settings di caption
    \DeclareCaptionFormat{custom} % un altro setting di caption
    {
        #1#2\textit{\small #3}
    }
    \captionsetup{format=custom}
\usepackage{soul} % per highlight migliore (\hl{} per sottolineare)
\usepackage{subcaption} % per più immagini in un unico ambiente figure alla volta

%//////////////////////////////////////////////////////
%/////////////(ri)definizioni varie////////////////////
%//////////////////////////////////////////////////////

% Per settare una dilatazione delle altezze delle righe
\renewcommand{\arraystretch}{1.15}

% Per Re ed Im molto più belli (secondo me)
\renewcommand{\Re}{\operatorname{Re}}
\renewcommand{\Im}{\operatorname{Im}}

% Per scrivere velocemente le derivate parziali
% comando: \pd{#1}{#2}
\newcommand{\pd}[2]{\dfrac{\partial #1}{\partial #2}}

% Per scrivere velocemente tra i capolari ("< >")
% comando: \lrangle{#1}
\newcommand{\lrangle}[1]{\langle #1 \rangle}

% Per scrivere i vari ±1/2 etc (SOLO NELLE EQUAZIONI)
\newcommand{\pmoh}{{\pm\frac{1}{2}}} % ±1/2
\newcommand{\mpoh}{{\mp\frac{1}{2}}} % ∓1/2
\newcommand{\poh}{{+\frac{1}{2}}} % +1/2
\newcommand{\moh}{{-\frac{1}{2}}} % -1/2

% \definecolor{bg}{rgb}{0.95,0.95,0.95}
% \definecolor{green}{rgb}{0,.66,0}
% \definecolor{white}{rgb}{1, 1, 1}